In this work, using Ruxanne we successfully mined 20 cross project bug fix patterns. We introduced these patterns in two groups: 12 general patterns and 8 BC-related patterns. Clippy is able to detect five of the BC-related patterns. Fundamentally, Clippy cannot detect some of them. For instance, we proposed the pattern \textit{Adding mutability} which Clippy cannot detect, simply because it cannot predict the next change that the developer is going to make. To the best of our knowledge, the rest of the patterns cannot be reported by the current linting tools. We therefore believe that our work introduces a new direction for IDE tools to help Rust developers.

In each group, we categorized the bug fix patterns based on the underlying program element. The patterns encompass a wide variety of program elements both in the general and BC-related patterns. This shows that our weighting scheme is not biased towards specific program elements. Also, our patterns are all cross-project fix patterns: each fix pattern has been seen at least in three different projects.

Furthermore, we expected to obtain some patterns associated with certain Rust-related features but our results did not include those patterns. Our speculation is that, since Rust has a strong compiler check before creating the program executable, the programmers usually get many errors, and they only push their code once it is free of compile errors. For instnace, we did not observe any patterns associated with \verb+move+ operator in Rust. This operator is used to transfer the ownership of values to a closure. In some cases (e.g. multithreading) not moving the ownership can create invalid references, something that Rust's compiler prevents. That is why we believe due to Rust's strong compiler check some patterns were hidden from our sight.

We believe our results can help researchers to create IDE tools for Rust. For instance, we discussed that \textit{Modifying the attributes of structs} is a common pattern that is observed in bug fix code changes. Assuming that the correct usage of attributes can be found other locations in the codebase~\citep{forrest2009genetic}, a code linter tool can be crafted to look for correct attribute list.

Similarly, we presented the pattern \textit{Dropping clone and adding borrowing}. As discussed, Clippy does not detect this pattern, and repetitive clones can be computationally expensive and might lead to hogging the CPU. Using state-of-the-art method for program repair, a tool can be designed to recognize this pattern and change variable cloning to a simple borrowing.

A general approach for code embedding is to extract paths with semantic information that serves better to our embedding goal. For instance, to design an embedding method for repairing Rust programs, the designer might want to disregard certain elements, such as moving ownerships to closure body---as they did not appear in the bug patterns we presented.

All of our parsing, path extraction, weighting scheme and clustering modules in addition to our final results can be found in our replication package for verification, resuability and further extension. The weighting can be modified to make the embedding focus on a specific set of elements for building different code embeddings.

\subsection{Threats to Validity}

In our work there are three main threats to internal validity: (1) We use a weighting scheme which is based on a heuristic to give more values to tree elements closer to the leaves. In addition to that, we also manually adjusted the weights based on our knowledge of Rust. As this approach is based on our own judgement, it does not guarantee achieving optimal clusters. Nevertheless, our weighting scheme can be readjusted to find different patterns, which is why we made our pipeline publicly available. (2) The collected commits are the ones that fix a bug in a previously pushed commit. We do not consider the bugs that occur during a commit and also fixed during the same commit. (3) Our code embedding approach is based on the frequency of observed program elements in ASTs, and we use DBSCAN as our clustering algorithm. Other program code embedding methods and clustering algorithms (e.g. SLINK) might output new clusters that our pipeline is unable to find. 

Threats to external validity can be described as: (1) The developers reporting the commits as bug fixing commits while they are not really fixing a functionality. Similarly, a commit message might not contain our target keywords while the commit is associated with bug fixing changes. (2) The selected projects might cause selection bias and therefore we might not have presented patterns that may exist in other projects. 
