In this work, we used Ruxanne to successfully mine 20 cross-project bug fix patterns. We presented these patterns in two groups: 12 general patterns and 8 BC-related patterns. Clippy is able to detect 5 of the 8 BC-related patterns. Fundamentally, Clippy cannot detect some of the remaining patterns. For instance, we proposed pattern \textit{Adding mutability}. In the Rust context, this change anticipates a future change to the code which will perform a mutation of a previously immutable value. Predicting the future is beyond Clippy's remit. To our knowledge, the remaining patterns (3 BC-related patterns and 12 general patterns) cannot be reported by current linting tools. 

In each group, we categorized the bug fix patterns based on their underlying program element. The patterns encompass a wide variety of program elements both in the general and BC-related groups. This supports a conclusion that our weighting scheme is not biased towards specific program elements. Also, our patterns are all cross-project fix patterns: each reported fix pattern has been seen at least in three different projects.

While we introduce patterns that reflect changes at different locations in the code, such as \textit{BC.ref.add}, which by borrowing a variable instead of taking ownership enables users to use the variable in subsequent locations, we acknowledge that it is possible that a large patch may be harder to cluster, because it has to resemble other patches. Additionally, we believe that any clustering-based approach will likely place larger diffs further apart for the same reason that ours does: there is just more going on. Nevertheless, there is nothing inherently penalizing a larger diff. Also, we point out that the related work by~\cite{pan2009toward} specifically excludes diffs with seven statements or more, while we allow larger diffs.

All of our parsing, path extraction, weighting scheme, and clustering modules, along with our final results, can be found in our replication package, which we provide for verification, reusability and further extension. Weightings can be modified to make the embedding focus on a specific set of elements for building different code embeddings.

\subsection{Patterns Present and Missing}
\label{subsec:patterns}

When presenting each pattern, we discussed actionability for the most important actions. Here, we discuss the implications of our most important patterns. \emph{In summary, program repair tools must understand semantics of attributes and should be performance-aware.} It may also be useful for tools to be able to automatically infer immutability. Here, we discuss the top ten most popular patterns (8 general patterns and 2 BC-related patterns).

The most common general pattern and the most popular pattern overall is modifying struct fields with 179 datapoints. Unfortunately, struct field modifications are neither interesting nor actionable; they are simply used to enable other changes. 

The most common borrow-checker-related pattern and second-most-common pattern overall is dropping \texttt{clone()}. This pattern is performance-related; when correctly applied, it has no semantic impact on the program, but reduces resource consumption. Rust is used in situations where performance is important, and the number of instances of this pattern confirms that developers do work on performance issues. This implies that \emph{program repair tools would benefit from being performance-aware}, rather than limiting themselves strictly to functional properties of the code.

Attribute modifications are interesting changes that account for the third- and ninth-most popular patterns. From a language design perspective, the fact that there are so many attribute changes marked as bug fixes illustrates the semantic importance of attributes in Rust, and \emph{implies that program repair must understand the effects of attributes}. Yet, different attributes have different, and often library-dependent, semantic meanings. Proposing specific repair techniques that reason about attributes is beyond the scope of this work, but our results suggest that they are important to develop.

The fourth- and fifth-most popular patterns, changing the type of a struct field or of a generic type parameter, are difficult to make general statements about; there are many unrelated idiosyncratic changes in these clusters. Such changes look the same, but have different purposes. One example we showed changed a \texttt{HashMap} to an \texttt{IndexMap}, preserving iteration order. That specific nondeterminism bug could be automatically repaired upon direction from the developer. Another change modified \texttt{usize} to \texttt{isize}, perhaps reflecting a requirements change to allow signed integers. IDEs could support refactorings that modify types. We find it difficult to propose further general work in this direction.

The sixth-most-popular pattern is the Rust-specific change of dropping mutability (i.e. making data immutable). Tools like ReImInfer by~\cite{HuangMDE2012} (for a Java extension), and techniques like the one proposed in~\cite{EyolfsonAbstractImmutability} for C++, aim to infer that data is immutable. In the Rust context, it is easier to make data immutable: the semantics for immutability are clear. Immutability is part of the core Rust language. Immutability escape hatches in C++ complicate issues, but in Rust, the escape hatch is the clearly-marked \texttt{unsafe} block. This pattern thus suggests that \emph{tools for inferring Rust immutability could be viable}.

We are unable to say anything meaningful about the seventh- and eighth-most popular patterns, \texttt{G.match.pattern} and \texttt{G.type.enum.variant}. They do not seem particularly helpful for automatic program repair.

The tenth-most-popular pattern, making a struct field \texttt{pub} and hence visible to external modules, suggests that tools to help developers manage visibility would be helpful. Public accessibility of methods and fields very much affects library compatibility and usability.

Conversely, we were surprised to find that Ruxanne reported no patterns associated with certain Rust-related features. For instance, we did not observe any patterns associated with the Rust \verb+move+ operator, which has been known to plague many beginning Rust programmers. \verb+move+ is used to transfer the ownership of values to a closure. In some cases (e.g. multithreading), not moving the ownership could create invalid references. Rust compiler checks prevent such errors. Our speculation is that, since Rust enforces strict compiler checks, programmers usually get many compile-time errors, but they only push their code once it is free of compile errors. We would not observe patterns which do not make it to the repository.

To compare the importance of bugs versus compilation errors: we consider bugs as potentially more critical. This is because a bug, if left unresolved, can remain dormant within the system and then inadvertently get deployed, whereas a program with compilation errors cannot be shipped by developers.

\vspace*{1em}
\fbox{\parbox{0.9\textwidth}{We summarize some learnings from our bug patterns:
\begin{itemize}
\item Rust program repair tools must understand the effects of attributes.
\item Program repair tools should be performance-aware.
\item Tools for inferring Rust immutability could be viable.
\end{itemize}
}}

\subsection{Usefulness of Rust Bug Patterns}

Having discussed the most frequent patterns and their implications on repair tools, we continue by discussing potential features for repair and refactoring tools in more detail. We believe our results can help with the development of program repair and refactoring tools for Rust. We have discussed the actionability of some of the general patterns and three BC-related patterns that are not detected by Clippy.

For instance, as we have discussed, for the common pattern of \textit{Modifying the attributes of structs}, a code linter tool could search for correct attribute lists within the codebase~\citep{forrest2009genetic}, leveraging the assumption that proper attribute usage can be found elsewhere in the code. Also, the linter can analyze the usages of the struct to determine the necessity of adding the attribute. Assuming that attributes cause additional code to be generated, removing unnecessary attributes not only optimizes the codebase but also eliminates unnecessary overhead and reduces binary size.

Similarly, we presented the pattern \textit{Dropping clone and adding borrowing}. As discussed, Clippy does not detect this pattern, and repetitive clones can be computationally expensive. Using state-of-the-art methods for program repair, one could design a tool to recognize this pattern and change variable cloning to simple borrowing.

We believe that our insights can greatly aid researchers in creating effective IDE tools tailored for Rust development. Moreover, the empirical frequency numbers associated with each pattern, highlighting its prevalence, can provide valuable guidance for the development of program repair tools.

% A general approach for code embedding is to extract paths with semantic information that better serve a given embedding goal. For instance, to design an embedding method for repairing Rust programs, the designer might want to disregard certain elements, such as moving ownership to closure bodies. Such elements may be irrelevant for repair because they did not appear in the bug patterns that we presented.


\subsection{Threats to Validity}

There are two main threats to the internal validity of our work: (1) A threat to internal validity is confounding, where changes to what shows up in the embedding are not due to changes in the code being embedded. In this case, the weighting scheme may contribute to confounding because we manually adjusted the weights. Nevertheless, our weighting scheme can be readjusted to find different patterns, which is why we made our pipeline publicly available. (2) Our code embedding approach is based on the frequency of observed program elements in ASTs, and we use DBSCAN as our clustering algorithm. Other code embedding methods and clustering algorithms (e.g. SLINK) might output new clusters that our pipeline is unable to find.

We discuss three threats to external validity: selection bias, incorrect commit/bug information reported by developers, and changes to the Rust programming language. (1) Selection bias is an issue because our benchmarks may differ systematically from the set of Rust programs in the world. Our selection of projects might be biased and therefore we might not have presented patterns that may exist in other projects. We aimed to mitigate this threat by choosing all of the most-starred projects, but this biases towards popular projects by definition. (2) Another threat to external validity can be the developers reporting commits as bug fixing commits while they are not really fixing functionality. Similarly, a commit message might not contain our target keywords while the commit is associated with bug fixing changes. This is simply an underlying assumption of our work and may cause us to miss some bugs; we do not believe it should be a systematic threat, and is shared by much other work on understanding bug fixes. (3) A third threat is that Rust is a relatively new programming language, and its syntax is prone to evolving over time. Changes in Rust's grammar, such as the introduction of new keywords like \verb+async/await+ in version 1.39, have the potential to introduce new bug categories or modify existing ones. Thus, one must consider language evolution as a potential external threat to the validity of our research findings; this threat can be mitigated by redoing our analysis in the future using our methodology with updated versions of our published artifacts.
