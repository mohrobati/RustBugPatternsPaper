%%%%%%%%%%%%%%%%%%%%%%% file template.tex %%%%%%%%%%%%%%%%%%%%%%%%%
%
% This is a general template file for the LaTeX package SVJour3
% for Springer journals.          Springer Heidelberg 2010/09/16
%
% Copy it to a new file with a new name and use it as the basis
% for your article. Delete % signs as needed.
%
% This template includes a few options for different layouts and
% content for various journals. Please consult a previous issue of
% your journal as needed.
%
%%%%%%%%%%%%%%%%%%%%%%%%%%%%%%%%%%%%%%%%%%%%%%%%%%%%%%%%%%%%%%%%%%%
%
% First comes an example EPS file -- just ignore it and
% proceed on the \documentclass line
% your LaTeX will extract the file if required
%
\RequirePackage{fix-cm}
%
%\documentclass{svjour3}                     % onecolumn (standard format)
%\documentclass[smallcondensed]{svjour3}     % onecolumn (ditto)
\documentclass[smallextended]{svjour3}       % onecolumn (second format)
%\documentclass[twocolumn]{svjour3}          % twocolumn
%
\smartqed  % flush right qed marks, e.g. at end of proof
%
\usepackage{graphicx}
%
% \usepackage{mathptmx}      % use Times fonts if available on your TeX system
%
% insert here the call for the packages your document requires
%\usepackage{latexsym}
% etc.
%
% please place your own definitions here and don't use \def but
% \newcommand{}{}
%
% Insert the name of "your journal" with
\journalname{Empirical Software Engineering}
%

\usepackage{listings, listings-rust}
\lstset{escapeinside={<@}{@>}}
\usepackage{tabularx}
\usepackage{array}
\usepackage{graphicx}
\usepackage{algorithm}
\usepackage{algpseudocode}
\usepackage[utf8]{inputenc}
\usepackage{tikz}
\usepackage{pgfplots}
\usetikzlibrary{graphs} 
\usetikzlibrary{matrix}
\usepackage{multirow}
% \usepackage[backend=bibtex, style=authoryear]{apa}
\usepackage{natbib}
% \addbibresource{bibtex.bib}
\usepackage{todonotes}
\usepackage{hyperref}
\usepackage{subfiles} % Best loaded last in the preamble
\usepackage{xcolor}
\usepackage{fontawesome5}
\newcommand{\RQBC}{RQ2}
\newcommand{\RQG}{RQ3}


\tikzset{ 
    table/.style={
        matrix of nodes,
        row sep=-\pgflinewidth,
        column sep=-\pgflinewidth,
        nodes={
            rectangle,
            draw=black,
            align=center
        },
        minimum height=1.5em,
        text depth=0.5ex,
        text height=2ex,
        nodes in empty cells,
        column 1/.style={
            nodes={text width=5em,font=\bfseries, fill=gray!20}
        },
        column 2/.style={
            nodes={text width=4em,font=\bfseries}
        },
        column 3/.style={
            nodes={text width=5em,font=\bfseries}
        },
        column 4/.style={
            nodes={text width=5em,font=\bfseries}
        },  
        column 5/.style={
            nodes={text width=8em,font=\bfseries}
        },  
        column 6/.style={
            nodes={text width=5em,font=\bfseries}
        },  
        column 7/.style={
            nodes={text width=5em,font=\bfseries}
        },
        column 8/.style={
            nodes={text width=9em,font=\bfseries}
        },
        column 9/.style={
            nodes={text width=3em,font=\bfseries}
        }
    }
}



\begin{document}

\title{A Study of Common Bug Fix Patterns in Rust
\thanks{Competing Interests: We have no competing interests and are funded by a Discovery Grant from Canada's Natural Science and Engineering Research Council.}
}
\subtitle{}
\definecolor{orcidlogocol}{rgb}{0.65, 0.807, 0.223}
\newcommand{\orcid}[1]{$\,$\href{https://orcid.org/#1}{\textcolor{orcidlogocol}{\faOrcid}}}

%\titlerunning{Short form of title}        % if too long for running head
\author{Mohammad Robati Shirzad \orcid{0000-0002-6297-8229}       \and
        Patrick~Lam \orcid{0000-0001-8278-5400} %etc.
}

%\authorrunning{Short form of author list} % if too long for running head

\institute{Mohammad Robati Shirzad --- \email{mrobatis@uwaterloo.ca} \\
           Patrick Lam --- \email{patrick.lam@uwaterloo.ca} \at
           University of Waterloo --- Address: 200 University Ave W, Waterloo, ON N2L 3G1 \\
%             \emph{Present address:} of F. Author  %  if needed
}

\date{Received: date / Accepted: date}
% The correct dates will be entered by the editor


\maketitle

\begin{abstract}
    Rust is a relatively new programming language which allows programmers to write programs that have low-level control over resources while still ensuring high-level safety guarantees (for programs written in safe Rust). Rust's ownership framework enables programs to meet these two seemingly-contradictory goals. The Rust compiler's Borrow-Checker component enforces the ownership framework requirements that ensure Rust's safety guarantees.

Rust is popular: as of 2022, it has ranked first, for the seventh consecutive year, in Stack Overflow's annual Developer Survey as the most-loved programming language. The number of Rust developers is growing as the need for faster and safer software increases. 

Yet, to the best of our knowledge, no research has sought to identify the most pervasive bug fix patterns within Rust programs. In this project, we introduce Ruxanne, a tool for analyzing and extracting fix patterns in Rust. Ruxanne implements a novel embedding of Rust code into fixed-sized vectors. Using Ruxanne, we mined the top 18 most-starred Rust projects in Github to discover the most common bug fix patterns committed to their repositories. We analyzed 87,726 code changes drawn from 57,214 commits across these 18 projects. After clustering the code changes, and conducting a manual analysis, we discovered 20 groups of cross-project bug fix patterns, which we categorize as (1) general patterns and (2) borrow-checker-related patterns. We describe each of these patterns and their implications to automated program repair.\todo{talk about the most popular pattern}
\keywords{Bug patterns \and Pattern mining \and Bug fix changes \and Rust}
% \PACS{PACS code1 \and PACS code2 \and more}
\subclass{68N01 \and 68N15} %% Computer Science : General and Programming Languages
\end{abstract}

\section{Introduction}
% rubber: depend introduction.tex
\subfile{introduction}

\section{Methodology}
% rubber: depend methodology.tex
\subfile{methodology}

\section{Results}
% rubber: depend results.tex
\subfile{results}

\section{Discussion}
% rubber: depend discussion.tex
\subfile{discussion}

\section{Related Work}
% rubber: depend related.tex
\subfile{related}

\section{Conclusion}
% rubber: depend conclusion.tex
\subfile{conclusion}


%\begin{acknowledgements}
%If you'd like to thank anyone, place your comments here
%and remove the percent signs.
%\end{acknowledgements}


% Authors must disclose all relationships or interests that 
% could have direct or potential influence or impart bias on 
% the work: 
%
% \section*{Conflict of interest}
%
% The authors declare that they have no conflict of interest.


% BibTeX users please use one of
\bibliographystyle{apa}      % basic style, author-year citations
% \bibliographystyle{spmpsci}      % mathematics and physical sciences
% \bibliographystyle{spphys}       % APS-like style for physics
\bibliography{refs}       % name your BibTeX data base
% \printbibliography[title={References}]  % name your BibTeX data base

% Non-BibTeX users please use
% \begin{thebibliography}{}
% %
% % and use \bibitem to create references. Consult the Instructions
% % for authors for reference list style.
% %
% \bibitem{RefJ}
% % Format for Journal Reference
% Author, Article title, Journal, Volume, page numbers (year)
% % Format for books
% \bibitem{RefB}
% Author, Book title, page numbers. Publisher, place (year)
% % etc
% \end{thebibliography}

\end{document}
% end of file template.tex

