Automated program repair aims to debug faulty source code without human involvement, sometimes using test cases to guide the repair. Before a repair tool modifies code, it will typically use a fault localization module to find fault locations. A fault localization module ranks program statements based on suspiciousness; one example is Tarantula by Jones and Harrold~\cite{jones2005empirical}. Suspiciousness is calculated using a statistical model which relies on the observation that buggy statements are executed mostly by failed test cases~\cite{naish2009spectral,xie2013theoretical}.

Researchers have proposed different methods for code repair, each of which can be said to follow a different mindset. Following the taxonomy propsed by~\cite{liu2018survey}, we briefly present a number of mindsets, and will discuss how having a set of predefined fix patterns can be beneficial for repair purposes.

\subsection{Search-Based Program Repair}
One line of research in program repair follows the competent programmer hypothesis~\cite{gopinath2014mutant}: syntactically, a faulty program is not that far away from its correct version. This hypothesis suggests the search-based program repair mindset. If the hypothesis holds, we can develop mutation operators (which change an expression or a statement) and apply them to a list of fault locations provided by the fault localization module. So, we loop through possible mutations; if the mutation of a statement does not cause all expected-successful test cases to succeed, we assume that the statement is correct and move on to the next fault location. Having a set of bug-producing patterns, like the ones we are proposing, will help researchers design targetted mutation operators that leverage domain specific insights.

Use of mutation operators in program repair became popular when Weimer et al.~\cite{forrest2009genetic,nguyen2009using} developed GenProg, a program repair tool that uses genetic programming. The main idea behind this tool is that statements follow certain patterns in a codebase. Therefore, if we find the correct version of a faulty statement somewhere in the program, we can use that version in place of the faulty original. This tool showed promising results as it managed to find patches without any additional annotations or human involvements.

However, Arcuri and Briand~\cite{arcuri2011practical} observed that GenProg mostly found the patch while carrying out random initialization---that is, before GenProg's evolution begins. Their tools TrpAutoRepair~\cite{qi2013efficient} and RSRepair~\cite{qi2014strength} advocate that test case prioritization is necessary while using genetic programming, as fitness evaluation is an expensive part of the repair pipeline.

Tan and Rovehoudhury~\cite{tan2015relifix} used mutation repairing for fixing regression bugs. They categorized common code changes in real-world regressions after studying 73 program evolution benchmarks; our approach also yields a set of common code changes, for Rust, using a different set of changes. Tan and Rovehoudhury used their categorization to design mutation operators, and chose them based on the faulty locations to repair the program.

\subsection{Pattern-based Program Repair and Mining Bug Fix Patterns}

% say why pattern-based repair is different from search-based repair -- isn't it sufficient to refer the reader to the survey?
The main intuition in pattern-based program repair methods is that the bug fixing modifications of the programs follow certain patterns. Thus, prior study of the common bug fix patterns is a crucial step for developing these repair tools. Pan et al.~\cite{pan2009toward} analyzed seven large-scale widely used Java projects and obtained 27 common bug fix patterns. Though they did the analysis on Java projects, their reported patterns were not specific to Java. Martinez and Monperrus~\cite{martinez2015mining,martinez2012mining} exploited the frequency of observed patterns to introduce a heuristic patch searching method. In their empirical evaluation, they concluded that the probabilistic values for repair actions can help with reducing the search space, hence creating a more effective repair tool.

Hanam et al.~\cite{hanam2016discovering} conducted similar research, but for finding JavaScript most pervasive bug fix patterns. They presented a large-scale study of the most common bug fix patterns by mining 105K commits from 134 server-side JavaScript projects. They used the DBSCAN clustering algorithm and divided bug fixing change types into 219 clusters, from which they extracted 13 pervasive cross-project bug fix patterns. 

Yang et al.~\cite{yang2022mining} proposed a mining approach to detect Python bug fix patterns via studying fine-grained fixing code changes. In their research, they also examined what portion of the bugs could be fixed using automated bug fixing approaches. Moreover, they evaluated the fix patterns in the wild and realized that 37\% of the buggy codes could be matched by the fix patterns they had found. 

We would characterize our work as the Rust version of what was done in~\cite{hanam2016discovering} and~\cite{yang2022mining}. Similar to~\cite{hanam2016discovering} we used DBSCAN to carry out the clustering, however, our novel program encoding approach differs from what they did. Also, analogous to~\cite{yang2022mining} we introduced patterns in two different categories, general and language specific (BC-related patterns).

\subsection{Code Patterns in Rust}

There have been some studies on common bug patterns in Rust, either for gaining insights from Rust programs or enhancing a tool's functionality. Nevertheless, to the best of our knowledge, our work is the first study that uses an automated mining and code analysis pipeline to find pervasive patterns in Rust open source projects.

Qin et al.~\cite{qin2020understanding} conducted the first empirical study on real-world Rust program behaviours. They manually inspected 850 unsafe code usages and 17 bugs across five open-source Rust projects, five Rust libraries, two online security databases, and the standard library of Rust. They analyzed the motivation behind unsafe code usage and removal, in addition to obtaining 70 memory-safety issues and 100 concurrency bugs. They also provided Rust programmers with some suggestions and insights to develop better Rust programs. At last, using the results of their analysis, they designed two bug detectors and provided recommendations for developing bug detector in the future.

Li et al.~\cite{li2021mirchecker} present MirChekcer, a fully automated bug detection framework for Rust. This framework works by carrying out static analysis on Rust's Mid-level Intermediate Representation (MIR). The tool exploits the insights obtained from observing existing bugs in Rust code bases. Through recording both numerical and symbolic information, the framework detects errors by using constraint solving techniques. The framework proves its practicability as it detected 33 new bugs including 16 memory-safety faults within 12 Rust crates.


