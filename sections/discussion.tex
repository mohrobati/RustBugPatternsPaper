\subsection{Threats to Validity}

In our work there are three main threats to internal validity: (1) We use a weighting scheme which is based on a heuristic to give more value to tree elements closer to the leaves. In addition to that, we also manually adjusted the weights of borrow checker related elements to respect ownership framework's role in Rust, though we recored their results separate from general results. As this approach is based on an intuition, it does not guarantee achieving optimal clusters. (2) The collected commits are the ones that fix a bug in a previously pushed commit. We do not consider the bugs that occur during a commit and also fixed during the same commit. (3) Our data modelling approach is based on the frequency of observed program elements in ASTs, and we use DBSCAN as our clustering algorithm. Other program data modelling methods and clustering algorithms (e.g. k-means, or SLINK) might output new clusters that our pipeline is unable to find. 

The threat to external validity can be the developers reporting the commits as bug fixing commits while they are not really fixing a functionality. Similarly, a commit message might contain our target keywords while the commit is not associated with bug fixing changes.
