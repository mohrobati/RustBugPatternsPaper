\documentclass{article}

\usepackage{cite}
\usepackage{graphicx}
\usepackage{hyperref}
\usepackage{todonotes}

\title{Response to Reviewer Comments}
\date{2023 November}
\author{Mohammad Robati Shirzad and Patrick Lam}
\begin{document}


\maketitle

Please find below our response to the reviews for our EMSE submission ``A Study of Common Bug Fix Patterns in Rust''.

\section{Reviewer 1}

\textit{I appreciate the authors' efforts in improving the paper, and most of my concerns have been addressed.}

\vspace*{1em}
Thanks!

\subsection{Reorganizing related work}

\textit{\textbf{R1:} I have observed that the related work now includes several subsections focusing on APR and patch correctness prediction. It would be beneficial to reorganize and consolidate these subsections into a single coherent section. Furthermore, the section on related work should encompass studies related to bug classification in programming languages with a longer history.}

\vspace*{1em} \noindent \textbf{Response:} We have reorganized the former Sections 5.1 through 5.3 into the new Section 5.1, adding a roadmap of that subsection at its beginning.

We also added a paragraph on bug classification at the beginning of Section 5.1. It adds three new references. The first is a 1975 study conducted by Enders et al. \cite{endres1975analysis}, which examines common error patterns in the DOS/VS operating system (written in the DOS Macro Assembler langage). The second study is by Knuth~\cite{knuth89:_error_tex}, surveying errors that he added in the development of \TeX. Finally, we added a 1998 study by Flanagan et al. \cite{flanagan1998new}, where they introduce a static debugger for Scheme (a Lisp dialect), centered around the error-prone operators in the language. Additionally, we moved the initial citation of Pan et al~\cite{pan2009toward} up to this paragraph; it studies bugs in seven large-scale Java programs.

\subsection{More discussion on the impact of the evolving Rust specification}

\textit{\textbf{R1:} Additionally, I would appreciate it if the authors could discuss the potential impact of the evolving Rust specification. Given that Rust is a novel programming language, it is subject to rapid grammatical changes, which in turn could influence bug characteristics and symptoms.}

\vspace*{1em} \noindent \textbf{Response:} Yes, the 2022 Rust roadmap\footnote{\url{https://blog.rust-lang.org/inside-rust/2022/04/04/lang-roadmap-2024.html}} includes future proposed changes to Rust. It is indeed a threat to external validity, and we have now discussed it in our threats to external validity (Section 4.3); this threat could be mitigated by refreshing the analysis results once new versions of Rust are broadly used by extant packages.

\section{Reviewer 2}

\textit{Thanks a lot for the information and clarification in the rebuttal and the new paper version. I would suggest the authors make the following changes in the next paper version.}

\subsection{More discussion on the size and type of the patches}

\textit{\textbf{R2:} Inspired by the last paragraph of Section 4.1, can authors discuss which of the following is more important for Rust in the next paper version: fixing bugs and fixing compilation errors?}

\vspace*{1em} \noindent \textbf{Response:} We added a new paragraph at the end of Section 4.1: we consider bugs as potentially more critical. This is because a bug, if left unresolved, can remain dormant within the system and then inadvertently get deployed, whereas a program with compilation errors cannot be shipped by developers.

\vspace*{1em} \noindent \textit{\textbf{R2:} Please discuss why identified patches are all small in the next paper version. Basically, please add the last paragraph of Section 2.2.4 of the rebuttal to the next paper version.}

\vspace*{1em} \noindent \textbf{Response:} Added the mentioned paragraph to the Discussion section of the manuscript (third paragraph of Section 4).

\subsection{Paraphrasing and providing clearer explanation}

\textit{\textbf{R2:} In Section 2.3.2, the authors use sentences from an existing paper directly (although quotes are added). I don't think this is a good way to write an academic paper. Please paraphrase these sentences.}

\vspace*{1em} \noindent \textbf{Response:} Paraphrased the three definitions that were directly quoted.

\vspace*{1em} \noindent \textit{\textbf{R2: } I still do not understand Figure 1. I suggest the authors label ASTDiff nodes with code to ease the explanation.}

\vspace*{1em} \noindent \textbf{Response:} We have added the labels {\tt \&tail} and {\tt head.clone()} at the relevant ASTDiff nodes on the tree on the left. The nonterminals themselves don't directly correspond to code---it would be the terminals which are not present in the paths. We have also added more text to the caption of Figure 1 and point at the ExprReference in Path 1 in the text just below the Figure.

\subsection{Minor issues}

\textit{\textbf{R2:} Line 21 on page 8: work around $\rightarrow$ workaround}

\vspace*{1em} \noindent \textbf{Response:} Fixed.

\vspace*{1em} \noindent \textit{\textbf{R2:} Line 49 on page 8: which path is colored in purple in figure 1?}

\vspace*{1em} \noindent \textbf{Response:} It is the rightmost path. We changed the color to blue so it is clearer and added a reference to ``Path 3'' to the text, to match with the Figure.

\small
\bibliographystyle{plain}
\bibliography{refs}

\end{document}
